\paragraph*{3.1 Описание практической части\\}

\hspace*{\parindent}Практическая часть лабораторной работы состоит из нескольких задач:\\

\begin{enumerate} 
\item[1.] Прочитать и постараться понять предложенный теоретический материал, представленный выше.
\item[2.] Собрать робот-манипулятор из наборов \textit{LEGO}.
\item[3.] Измерить и рассчитать для него параметры Денавита - Хартенберга.
\item[4.] По полученным параметрам расписать ПЗК системы.
\item[5.] А также ОЗК системы.
\item[6.] Проверить правильность своих расчетов посредством подстановки прямой и обратной задач в пакете \textit{Scilab}.
\item[7.] Написать программу, позволяющею по входным параметрам (координатам в обобщенной системе) перемещать рабочий инструмент (хват) в заданное положение и определять свои параметры в текущий момент (углы вращения по входным координатам)
\item[8.] Написать программу, позволяющею по входным параметрам (углам вращения) перемещать рабочий инструмент (хват) в заданное положение и определять свои параметры в текущий момент (координаты обобщенной системы по входным углам вращения).
\item[9.] Также в пакете \textit{Scilab} построить траектории движения и сравнить с имеющийся на готовом роботе. 
 \end{enumerate}